\section{Техническое задание}

Техническое задание на разработку тонкого клиента на базе микрокомпьютера, шифр «Клиент»

\begin{easylist}[articletoc]
\ListProperties(Progressive*=3ex,Hide4=4)
@ Цель выполнения ТЗ, наименование
@@ Целью выполнения ТЗ шифр «Клиент» является разработка тонкого клиента с корпусом,
изготовленным с помощью аддитивных технологий
@@ Наименование изделия: Тонкий клиент
@@ Краткое наименование изделия: изделие «Клиент»
@ Тактико-технические требования к изделию
@@ Состав изделия «Клиент»
@@@ Изделие «Клиент» должно включать одно или более твердых тел, объединенных в корпус
@@@ В состав корпуса предполагается установка платы микрокомпьютера Raspberry~Pi 3B c
microSD-картой
@@ Конструктивные требования
@@@ Масса изделия не должна превышать 300 г
@@@ Материалы, возможные для использования: 
@@@@ — Пластик на основе полилактида (PLA)
@@@@ — ABS-пластик
@@@@ — Пластик PET-G
@@ Требования назначения
@@@ В изделии должен быть доступ к разъемам микрокомпьютера Raspberry~Pi:
@@@@ — USB-A
@@@@ — HDMI
@@@@ — Ethernet (RJ-45)
@@@@ — Разъем питания microUSB
@@@ Должно быть предусмотрено отверстие для подключения и извлечения microSD-карты
@@@ Электропитание изделия «Клиент» должно осуществляться от сети
переменного тока напряжением 220 В ± 10\%, частотой 50 Гц ± 10\%.
@@ Требования живучести и стойкости к внешним воздействиям
@@ Требования транспортабельности
@@@ Упаковка изделия «Клиент» должна допускать его транспортирование
в условиях группы Жт по ГОСТ В 9.001-72 при условии исключения воздействия
на упаковку атмосферных осадков. В транспортном средстве упаковка с
изделием «Корпус РЭА» должна быть закреплена.
@@@ Условия транспортирования изделия «Клиент» в части воздействия
климатических факторов должны соответствовать условиям его хранения в
неотапливаемых помещениях по ГОСТ В 9.003-80 при температуре от минус
40°С до 50°С.
\end{easylist}

