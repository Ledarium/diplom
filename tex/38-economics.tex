\subsection{Экономическая эффективность}

Для оценки экономической эффективности нужно сравнить затраты на аппаратное и
программное обеспечение, используемое в проекте. Цены на компьютерные комплектующие 
взяты с сервиса агрегации цен E-katalog. Цены на ПО взяты с сайтов официальных
дистрибуторов ПО (или их представителей в Российской Федерации).

Тут табличка цен на ПК vs цены на сервер+клиенты, график показывающий эффективность

Тут табличка с лицензиями виндовс

Тут табличка с лицензиями солида

Таким образом, можно сделать вывод о экономической целесообразности реализации данного
проекта. По сравнению с используемой на кафедре системой из полноценных компьютеров
(толстых клиентов), система тонких клиентов дает возможность значительно снизить затраты
при подключении требуемого количества клиентов, получая более высокую производительность
рабочих мест.
При подключении 25 клиентов, что является максимально возможным количеством
пользователей для используемой лицензии Windows Server Essentials, экономия средств на
программное обеспечение, составляет XX\%, на аппаратное — XX\%. Стоит отметить, что при
необходимости дальнейшей модернизации, достаточно будет обновить аппаратное обеспечение
только серверной части, что также уменьшает затраты на оборудование.
