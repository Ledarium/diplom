\subsection{Терминальные системы}
Первые компьютеры 50-х годов предназначались для очень небольшого числа избранных
пользователей. Такие компьютеры не были предназначены для интерактивной работы
пользователя, а применялись в режиме пакетной обработки.  Задания нескольких
пользователей группировались в пакет, который принимался на выполнение. Распечатанные
результаты пользователи получали обычно только на следующий день.

По мере удешевления процессоров в начале 60-х годов появились новые способы организации
вычислительного процесса, которые позволили учесть интересы пользователей.  Начали
развиваться интерактивные многотерминальные системы разделения времени.  В таких
системах каждый пользователь получал собственный терминал, с помощью которого он мог
вести диалог с компьютером.
Действительно, рядовой пользователь работу за терминалом мэйнфрейма воспринимал примерно
так же, как сейчас он воспринимает работу за подключенным к сети персональным
компьютером. Пользователь мог получить доступ к общим файлам и периферийным устройствам,
при этом у него поддерживалась полная иллюзия единоличного владения компьютером, так как
он мог запустить нужную ему программу в любой момент и почти сразу же получить
результат. \cite{olifer}

Дальнейшее развитие информационных технологий и удешвление персональных компьютеров
привело к падению релевантности терминальных систем. Однако, в настоящее время тип
организации сетей, в основе которого лежат терминалы, называемые \say{тонкими клиентами}
(в противовес \say{толстым клиентам} — обычным ПК), вновь набирает популярность. Из-за
увеличения числа ПК в сетях их обслуживание становится все более сложным. Поэтому тонкие
клиенты стали снова широко использоваться.
