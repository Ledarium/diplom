\addcontentsline{toc}{chapter}{Определения и сокращения}
\chapter*{Определения и сокращения}

В настоящей выпускной квалификационной работе используются следующие термины, сокращения
и обозначения с соответствующими определениями:

%\renewcommand{\descriptionlabel}[1]{\hspace{\labelsep}{#1}}

\noindent
	\begin{desclist}{}{ \rm\hfill —}[ПО]
		\item[БД] база данных
        \item[Дистрибутив] форма распространения программного обеспечения
        \item[Локальная вычислительная сеть (ЛВС)] частная сеть, размещенная, как
            правило, в одном здании или на территории одной организации
		\item[ОС] операционная система
		\item[ПАК] программно-аппаратный комплекс
		\item[ПК] персональный компьютер
		\item[ПО] программное обеспечение
        \item[САПР] система автоматизированного проектирования
		\item[ТК] тонкий клиент, терминал, устройство терминального доступа
		\item[Ethernet] семейство технологий пакетной передачи данных между
	устройствами для компьютерных и промышленных сетей. Ethernet в основном
	описывается стандартами IEEE группы 802.3
        \item[Linux] семейство Unix-подобных операционных систем на базе ядра Linux,
            включающих тот или иной набор утилит и программ, и, возможно,
            другие компоненты. 
        \item[Local Area Network (LAN)] локальная вычислительная сеть
        \item[Remote Desktop Protocol (RDP)] протокол удаленного доступа, используемый в
            ОС семейства Windows.
	\end{desclist}
