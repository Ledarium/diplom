\subsection{Постановка задачи на разработку}

На момент разработки на кафедре КПРС имеется ЛВС, построенная по классической
одноранговой топологии \say{звезда}. Каждое рабочее место, подключенное к сети, является
полноценным \say{толстым клиентом}. Сеть используется только для предоставления доступа
к Интернет. Прочие сетевые технологии (такие как общий файловый сервер) практически не
используются. Исходя из этого, можно выделить задачи на разработку:

\begin{enumerate}
    \item Необходимо модернизировать сеть кафедры для обеспечения возможности
        совместного доступа к ресурсам сети и создания выделенного компьютера,
        выполняющего роль сервера рабочих столов.
    \item Разработать экономически целесообразное устройство, используемое в качестве
        тонкого клиента.
    \item Заменить часть рабочих мест на тонкие клиенты.
\end{enumerate}

Для выполнения этих задач было принято решение создать на базе существующей сети
программно-аппаратный комплекс, включающий в себя сервер терминалов, тонкие клиенты и
сеть, соединяющую их для совместного доступа к ресурсам. Требования к ПАК приведены в
таблице~\ref{tab:reqs}.

\begin{table}[h]
    \centering
    \caption{Технические требования к комплексу}
    \label{tab:reqs}
    \begin{tabu}to \linewidth{Xr}
        \toprule
        Технология канального уровня & Ethernet \\
        Количество подключенных рабочих мест, от & 3 \\
        Пропускная способность, Гбит & 10 \\
        \midrule
        Возможность многопользовательского доступа & SolidWorks \\
                                                   & Altium Designer \\
                                                   & OrCAD PCB \\
                                                   & и т.п. \\
        \midrule
        Масштабируемость & \\
        Экономическая эффективность & \\
        Экологичность и энергоэффективность & \\
        Возможность постепенной модернизации & \\
        \multicolumn{2}{l}{Максимальное использование имеющегося аппаратного
        обеспечения} \\
        \bottomrule
    \end{tabu}
\end{table}
