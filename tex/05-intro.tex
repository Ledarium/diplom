\addcontentsline{toc}{chapter}{Введение}
\chapter*{Введение}

С развитием вычислительной техники системные требования программного обеспечения
становятся все более высокими. За последнее время производительность компьютеров
значительно выросла, поэтому старые устройства уже не справляются с современными
программами, особенно с требовательными САПР, используемыми в учебном процессе
на кафедре КПРС.

Актуальность работы состоит в том, что новые версии таких САПР, как SolidWorks 2019 и
Altium Designer 19, неудовлетворительно работают на компьютерах кафедры.
При модернизации компьютерных классов есть
возможность использовать подход к организации сети, основанный на тонких клиентах —
терминальных станциях, которые подключаются к производительному серверу, на котором
производятся вычисления.

Целью выпускной квалификационной работы является разработка общевычислительного
многопользовательского программно-аппаратного комплекса (ПАК) на основе тонких клиентов.

Объектом исследования является вычислительный комплекс кафедры.

Предмет исследования — модернизация вычислительного комплекса.

Для достижения заданной цели были поставлены следующие задачи:

\begin{enumerate}
    \item Рассмотреть технологии организации вычислительных сетей
    \item Разработать проект программно-аппаратного комплекса
    \item Установить и настроить комплекс
    \item Проанализировать результаты работы
\end{enumerate}

Практическая значимость работы состоит в том, что вычислительный комплекс кафедры может
быть модернизирован в соответствии с выдвигаемыми в работе предложениям, по прописанным
пошаговым инструкциям.

В первой главе рассматриваются основные технологии организации вычислительных сетей.

Во второй главе разрабатывается проект ПАК, выбирается программное и аппаратное
обеспечение.

В третьей главе описывается процесс установки и настройки комплекса, разрабатывается
корпус для изделия.

В четвертой главе анализируются результаты работы, исследуется производительность и
экономическая эффективность, проводится анализ на устойчивость к внешним воздействиям.

В заключении формулируются выводы и предложения по использованию результатов работы.
